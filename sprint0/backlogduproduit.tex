\section{Backlog De Produit}
\noindent
\large
Le backlog de produit est une liste de fonctionnalités à réaliser. Ces fonctionalités sont exprimées sous formes des besoins et sont priorisées par le Product Owner ce qui permet d'établir un ordre de réalisation à respecter. \\
Notre backlog est composé de trois colonnes: \\
\textbf{ID: } C'est l'identifiant du scénario \\
\textbf{Fonctionnalité: } Permet de mieux ordonner les scénarios. \\
\textbf{Scénario: } Comporte la description des scénarios suivant le forme << En tant que ... Je veux >> \\

\begin{table}[H]
	\centering
	\begin{tabular}{|c|c|p{10cm}|}
		\hline
		\rowcolor{blue!20}
		\textbf{ID} & \textbf{Fonctionnalité}                 & \textbf{Scénario}                                                                                                                      \\ \hline
		1           & Recherche en Arabe en dialecte Tunisien & En tant qu'un client, visiteur, je veux rechercher un ou plusieurs produits dans la langue Tunisienne.                                 \\ \hline
		2           & Recherche en Arabe Classique            & En tant qu'un client, visiteur, je veux rechercher un ou plusieurs produits en Arabe Classique.                                        \\ \hline
		3           & Recherche en Français                   & En tant qu'un client, visiteur, je veux rechercher un ou plusieurs produits en Français.                                               \\ \hline
		4           & Suggestion des produits                 & En tant qu'un client, visteur, si le produit exacte que je cherche n'existe pas, je veux voir des suggestions des produits similaires. \\ \hline

		5           & Modifier paramètres de recherche                 & En tant qu'un administrateur, je veux modifier mes paramètres de recherche des produits. \\ \hline

		6           & Consulter tableau de bord               & En tant qu'un administrateur, je veux s'authentifier et accéder à mon tableau de bord.                                                 \\  \hline
	\end{tabular}
	\caption{Table des fonctionnalités et scénarios.}
\end{table}

\section{Plannification des sprints}
\noindent
Une fois que le Product Backlog a été établi, la réunion de planification peut être
tenue. Le but de cette réunion est de préparer le plan de travail. Le résultat de cette réunion est
le choix des durées des Sprints. La liste suivante détaille la planification des Sprints: \\

\newpage
\noindent
\textbf{Sprint 1:}
\begin{enumerate}
	\item Rechercher des produits en Français
	\item Voir suggestions
\end{enumerate}

\noindent
\textbf{Sprint 2:}
\begin{enumerate}
	\item Rechercher des produits en Arabe Traditionnel
	\item Rechercher des produits en Arabe en dialecte Tunisien
	\item Voir suggestions
\end{enumerate}

\noindent
\textbf{Sprint 3:}
\begin{enumerate}
	\item Modifier paramètres de recherche
	\item Consulter tableau de bord
\end{enumerate}


% \begin{table}[H]
% 	\centering
% 	\Large
% 	\rowcolors{2}{white}{white} % To reset alternate colors if previously set
% 	\begin{tabular}{|p{10cm}|p{4cm}|p{4cm}|}
% 		\hline
% 		\rowcolor{blue!50} \textcolor{white}{Sprint 1}              & \textcolor{white}{Sprint 2} & \textcolor{white}{Sprint3} \\
% 		\hline
% 		\begin{enumerate}
% 			\item Rechercher des produits en Français
% 			\item Voir suggestions
% 		\end{enumerate}                   & Auto-incrément              & Clé primaire                                         \\ \hline
% 		\begin{enumerate}
% 			\item Rechercher des produits en Arabe Traditionnel
% 			\item Rechercher des produits en Arabe en dialecte Tunisien
% 			\item Voir suggestions
% 		\end{enumerate} & String                      & Non nul                                                             \\ \hline
% 		password                                                    & String                      & Non nul                    \\ \hline
% 		role                                                        & String                      & Non nul                    \\ \hline
% 	\end{tabular}
% 	\caption{Tableau de plannifiation des sprints}
% 	\label{tab:plansprints}
% \end{table}