\section{Backlog De Produit}
\noindent
\large
Le backlog de produit est une liste de fonctionnalités à réaliser. Ces fonctionalités sont exprimées sous formes des besoins et sont priorisées par le Product Owner ce qui permet d'établir un ordre de réalisation à respecter. \\
Notre backlog est composé de trois colonnes: \\
\textbf{ID: } C'est l'identifiant du scénario \\
\textbf{Fonctionnalité: } Permet de mieux ordonner les scénarios. \\
\textbf{Scénario: } Comporte la description des scénarios suivant le forme << En tant que ... Je veux >> \\

\begin{table}[H]
\centering
\begin{tabular}{|c|c|p{10cm}|}
\hline
\rowcolor{blue!20}
\textbf{ID} & \textbf{Fonctionnalité} & \textbf{Scénario} \\ \hline
1 & Recherche en Arabe Tunisienne  & En tant qu'un client, je veux rechercher un ou plusieurs produits dans la langue Tunisienne. \\ \hline
2 & Recherche en Arabe Classique & En tant qu'un client, je veux rechercher un ou plusieurs produits en Arabe Classique. \\ \hline
3 & Recherche en Français & En tant qu'un client, je veux rechercher un ou plusieurs produits en Français. \\ \hline
4 & Vitesse de recherche & En tant qu'un client, je veux avoir une vitesse de recherche rapide. \\ \hline
5 & Suggestion des produits & En tant qu'un client, si le produit que je cherche n'existe pas, je veux avoir des suggestions des produits similaires. \\ \hline
\end{tabular}
\caption{Table des fonctionnalités et scénarios.}
\end{table}