\section{Identification des besoins}
\subsection{Les besoins fonctionnels}
\noindent
Les besoins fonctionnels sont les fonctionnalités que le systéme doit livrer aux utilisateurs.
L’outil n’est considéré comme opérationnel que si sa disponibilité fonctionnelle est garantie.
Dans le cas du notre systéme, ces besoins se concentrent sur:

\begin{itemize}
    \item \small\textbf{Vitesse de recherche: } {Améliorer les vitesses de recherche au maximum afin de renvoyer des résultats précis au client.}
    
    \item \small\textbf{Recherche en Français: } {Permettre au client de rechercher les produits dans le langage française.}

    \item \small\textbf{Recherche en arabe tunisienne: } {Permettre au client de rechercher les produits dans le langage arabe tunisienne.}

    \item \small\textbf{Recherche en arabe traditionnel: } {Permettre au client de rechercher les produits dans le langage arabe classique.}
    
    \item \small\textbf{Suggestion des produits: } {Si le produit recherché par le client n'existe pas, le système tentera de suggérer des produits similaires en prenant le contexte du terme de recherche.}
\end{itemize}

\newpage
\subsection{Les besoins non fonctionnels}
\noindent
Les exigences non fonctionnels décrivent les objectifs liés aux performances du système et d'autres aspects cruciaux du système qui ne sont pas directement liés à ses fonctionnalités spécifiques. Ils définissent les critères de qualité que le système doit respecter pour répondre aux attentes des utilisateurs. Notre système doit répondre aux exigences non fonctionnelles suivantes:

\begin{itemize}
    \item \small\textbf{La Fiabilité: } L'application doit être fonctionnelle sans détection des erreurs afin de satisfaire les besoins du client.

    \item \small\textbf{La Sécurité: } Vu que l'application contient des données confidentielles, tout accés aux produits doit être protégés par un privilége d'accées.

     \item \small\textbf{La Disponibilité: } Les services offerts par notre application sont disponibles pendant les 24
     heures et durant toute la semaine.

     \item \small\textbf{La Performance: } L'application doit être rapide et robuste (Vitesse de réponse rapide et précision des résultats lors de recherche des produits dans les langues différentes).
\end{itemize}

\newpage