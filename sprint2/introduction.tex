\chapter{Étude et réalisation du Sprint 2}
\section{Introduction}
\noindent
Aprés avoir terminé le Sprint 1 du chapitre 4 qui tratait la recherche des produits dans la langue Française, nous entamons maintenant le Sprint 2 qui va aborder la recherche en Arabe en dialecte Tunisien, et ensuine l'Arabe Traditionnel.

\section{Backlog du Sprint 2}
\begin{table}[H]
	\centering

	\begin{tabularx}{\textwidth}{|c|X|c|c|}
		\hline
		\rowcolor{blue!20}
		\textbf{ID} & \textbf{Scénario}                                                                                     & \textbf{Priorité} & \textbf{Complexité} \\ \hline
		1           & En tant qu'un client, je veux saisir ma terme de recherche en Arabe en dialecte Tunisien pour chercher le(s) produit(s) & 1                 & 7                  \\ \hline

	2           & En tant qu'un client, je veux saisir ma terme de recherche en Arabe Traditionnel pour chercher le(s) produit(s) & 2                 & 6 \\ \hline
	\end{tabularx}
	\caption{Backlog du Sprint 2}
	\label{tab:sprint2}
\end{table}

\section{Spécification fonctionnelle}
\noindent
Dans cette partie, on va expliquer les différentes fonctionnelités du Sprint 2 à travers le diagramme de cas d'utilisation. Puis en va exposer les différents scénarios de notre cas d'utilisation à travers des descriptions textuelles.

\section{Diagramme de cas d'utilisation général}
\noindent
La figure ~\ref{fig:recherchearabe} illustre le diagramme de cas d'utilisation générale de notre deuxième sprint, qui est la recherche en arabe traditionnelle et arabe en dialecte tunisien.

\begin{figure}[H]
	\centering
	\includegraphics[width=1\textwidth]{logos/cusprint2.png}
	\caption{Diagramme de cas d'utilisation générale du Sprint 2}
	\label{fig:recherchearabe}
\end{figure}

\section{Description textuelle des cas d’utilisations}
Une fois les divers cas d'utilisation présentés, nous examinerons de plus près certains
d'entre eux en fournissant la description textuelle de certains d'entre eux.

\newpage
\subsection{Description textuelle du CU « Rechercher produit en arabe traditionnel »}
\noindent
\textbf{Titre:} Rechercher produit en arabe traditionnel \\
\textbf{Résumé:} Le client saisit son terme de recherche en arabe traditionnel, en cliquant sur le boutton pour rechercher le(s) produit(s) qu'il veut chercher. \\
\textbf{Acteur Principal:} Client \\
\textbf{Précondition:} \begin{enumerate}
	\item Le client (ou le visiteur) est sur la page de recherche
	\item Le client (ou le visiteur) a saisi son terme de recherche en arabe traditionnel
	\item Le client (ou le visiteur) a cliqué sur "Rechercher"
\end{enumerate}
\textbf{Postcondition:} Le(s) produit(s) que le client (ou le visiteur) cherche est renvoyé, s'il n'existe pas, le systéme renvoie des produits similaires comme suggestion. \\
\textbf{Scénario de base: }
\begin{enumerate}
	\item Le client saisit son terme de recherche.
	\item Le client clique sur le boutton "Rechercher"
	\item Le système prend le terme de recherche, en vérifiant que c'est en arabe traditionnel, et le traduit en français.
	\item Le système prend ce terme de recherche, et performe les étapes nécessaires pour la convertir en vecteur.
	\item Le système compare ce vecteur contre les vecteurs dans Elasticsearch.
	\item Le système renvoie les produits.
\end{enumerate}

\newpage
\textbf{Scénario alternatifs : }
\begin{enumerate}
	\item Le terme de recherche est vide:
	      \begin{enumerate}
		      \item Le système affiche un message d'erreur informant le client que le terme de recherche est requis.
		      \item Retour à l'étape 1 du scénario de base.
	      \end{enumerate}
	\item Le terme de recherche n'est pas en arabe traditionnel:
	      \begin{enumerate}
		      \item Le système suppose que le terme recherché est en drançais.
		      \item Passer à la 4ème étape des scénarios de base.
	      \end{enumerate}
	\item Le(s) produit(s) que le client cherche n'existe pas.
	      \begin{enumerate}
		      \item Le système essaie de renvoyer les produits les plus similaires comme des suggestions.
		      \item Retour à l'étape 1 du scénario de base.
	      \end{enumerate}
\end{enumerate}

\subsection{Description textuelle du CU « Rechercher produit en arabe en dialecte tunisien »}
\noindent
\textbf{Titre:} Rechercher produit en arabe en dialecte tunisien \\
\textbf{Résumé:} Le client(ou le visiteur) saisit son terme de recherche en arabe en dialecte tunisien, en cliquant sur le boutton pour rechercher le(s) produit(s) qu'il veut chercher. \\
\textbf{Acteur Principal:} Client \\
\textbf{Précondition:} \begin{enumerate}
	\item Le client (ou le visiteur) est sur la page de recherche
	\item Le client (ou le visiteur) a saisi son terme de recherche en arabe en dialecte tunisien
	\item Le client (ou le visiteur) a cliqué sur "Rechercher"
\end{enumerate}
\textbf{Postcondition:} Le(s) produit(s) que le client cherche est renvoyé, s'il n'existe pas, le système renvoie des produits similaires comme suggestion. \\
\textbf{Scénario de base: }
\begin{enumerate}
	\item Le client saisit son terme de recherche en arabe en dialecte tunisien.
	\item Le client clique sur la boutton "Rechercher"
	\item Le système prend le terme de recherche, en vérifiant que c'est en arabe en dialecte Tunisien, et le traduit en Français.
	\item Le système prend ce terme de recherche, et performe les étapes nécessaires pour la convertir en vecteur.
	\item Le système compare ce vecteur contre les vecteurs dans Elasticsearch.
	\item Le système renvoie les produits.
\end{enumerate}

\textbf{Scénario alternatifs : }
\begin{enumerate}
	\item Le terme de recherche est vide:
	      \begin{enumerate}
		      \item Le système affiche un message d'erreur informant le client que le terme de recherche est requis.
		      \item Retour à l'étape 1 du scénario de base.
	      \end{enumerate}
	\item Le terme de recherche n'est pas en arabe en dialecte tunisien:
	      \begin{enumerate}
		      \item Le système suppose que le terme recherché est en français.
		      \item Passer à la 4ème étape des scénarios de base.
	      \end{enumerate}

	\item Le(s) produit(s) que le client cherche n'existe pas.
	      \begin{enumerate}
		      \item Le système essaie de renvoyer les produits les plus similaires comme des suggestions.
		      \item Retour à l'étape 1 du scénario de base.
	      \end{enumerate}
\end{enumerate}

\subsection{Description textuelle du CU << Voir suggestions >>}
\noindent
\textbf{Titre:} Voir suggestions \\
\textbf{Résumé:} Après la recherche du produit, le systéme essaie de suggérer au client des produits similaires. \\
\textbf{Acteur Principal:} Client \\
\textbf{Précondition:} \begin{enumerate}
	\item Le client est authentifié.
	\item Le client a déja cherché un produit.
\end{enumerate}
\textbf{Postcondition:} Le(s) produit(s) que le client cherche est renvoyé, et le systéme renvoie des produits similaires comme suggestions. \\
\textbf{Scénario de base: }
\begin{enumerate}
	\item Le client cherche un produit.
	\item Le système cherche le produit.
	\item Le système renvoie les produits ainsi que des suggestions des produits similaires.
\end{enumerate}

\textbf{Scénario alternatifs : }
\begin{enumerate}
	\item Le(s) produit(s) que le client cherche n'existe pas.
	      \begin{enumerate}
		      \item Le système essaie de renvoyer les produits les plus similaires comme des suggestions.
		      \item Retour à l'étape 1 du scénario de base.
	      \end{enumerate}
\end{enumerate}
\section{Conception}
\noindent
Dans cette partie, nous allons présenter le diagramme de séquence correspondant à notre diagramme cas d'utilisation précédemment présentés dans les descriptions textuelles pour le deuxième Sprint.

\subsection{Diagramme de séquence détaillé}
\noindent
Nous avons regroupé les deux cas d'utilisation << Rechercher produits en Arabe traditionnel >> et << Rechercher produits en Arabe en dialecte Tunisien >> dans un seul diagramme de séquence présenté dans la figure ~\ref{fig:diagseqsprint2}.

\begin{figure}[H]
	\centering
	\includegraphics[width=1\textwidth]{logos/seqsprint2.png}
	\caption{Diagramme de séquence des cas d’utilisations « Rechercher produits en Arabe Traditionnel » et << Rechercher produits en Arabe en dialecte Tunisien >>}
	\label{fig:diagseqsprint2}
\end{figure}

\section{Réalisation}
\noindent
Cette partie est consacrée à la présentation des étapes nécessaires pour réaliser le travail nécessaire pour satisfaire notres cas d'utilisations, qui consiste à permettre le client ou le visiteur à rechercher notres produits en Arabe traditionnel et Arabe en dialecte tunisien tout en améliorant l'expérience de recherche en utilisant la traduction et la recherche vectorielle via Elasticsearch.

\newpage
\subsection{Les premières approches}
\noindent
Puisque notre modèle actuel n'est formé que sur les langues latines, nous avons eu l'idée de l'entraîner à la fois sur l'arabe tunisien et l'arabe traditionnel, mais un certain nombre de limitations nous empêchaient de le faire:

\begin{enumerate}
	\item Absence totale de jeux de données sur la langue arabe en dialecte Tunisien.
	\item Absence totale de jeux de données sur la langue Arabe Traditionnel.
	\item Le processus de l'entraînement nécessite une machine beaucoup plus puissante et une période de temps très longue afin de traiter correctement les 2 langues que nous avons citées.
\end{enumerate}

\noindent
Nous avons également testé avec le « Fine-Tuning » pour entraîner notre modéle, nous définissons ce processus comme suit: \\
\textit{Le << Fine-Tuning >> consiste à prendre un modèle d'apprentissage automatique pré-entraîné et à le former davantage sur un ensemble de données plus petit et ciblé. L'objectif du réglage fin est de conserver les capacités d'origine d'un modèle pré-entraîné tout en l'adaptant à des cas d'utilisation plus spécialisés.} \\ \citetitle{techtarget:finetuning} (\cite{techtarget:finetuning})

\noindent
Mais ce processus nécessitait un ensemble de données beaucoup plus volumineux que celui que nous avions préparé et prenait trop de temps, nous avons donc décidé d'adopter l'approche mentionnée ci-dessous.

\newpage
\subsection{Création de notre propre classe traducteur}
\noindent
Avec les limitations de l'approche << Fine-Tuning >> que nous avons mentionnée précédemment, nous avons décidé de créer notre propre traducteur qui va:
\begin{enumerate}
	\item Vérifier si le terme de recherche est en Arabe traditionnel, et le traduire en Français à partir de l'API Google Traduction.
	\item Vérifier si le terme de recherche est en Arabe en dialecte Tunisien à partir de notre propre dictionnaire des mots Tunisiens et leur equivalent en Français, si il n'y a pas d'équivalent exacte, il essaie de trouver l'équivalent en calculant un pourcentage, et s'il n'y a pas d'équivalent même après avoir calculé le pourcentage, le mot reste tel quel en supposant qu'il soit en Français
\end{enumerate}

\subsubsection{La création du dictionnaire}
\noindent
D'abord, nous commençons par préparer notre classe, que nous avons nommée << TunisianTranslator >> et initialiser un attribut privé, qui est notre dictionnaire. La figure ~\ref{fig:dictionary} montre le code nécessaire pour cette étape.

\begin{figure}[H]
	\centering
	\includegraphics[width=0.6\textwidth]{logos/dictionary.png}
	\caption{Code d'initialisation de classe et du dictionnaire}
	\label{fig:dictionary}
\end{figure}

\noindent
Ensuite, nous continuons en définissant notre première méthode dans la classe qui est une méthode privée nommée \texttt{\_\_is\_latin}. Cette méthode est conçue pour vérifier si une chaîne donnée s contient uniquement des caractères des blocs Unicode Basic Latin et Latin-1 Supplement, elle renvoie True si tous les caractères de la chaîne << s >> se trouvent dans la plage Unicode U+0000 à U+00FF, ce qui correspond aux blocs Basic Latin et Latin-1 Supplement. Si un caractère se situe en dehors de cette plage, la méthode renvoie False.

\subsubsection{Analyse des expressions régulières}
\noindent
L'expression régulière utilisée dans notre méthode est :
\Large\[ [^{\backslash u0000-\backslash u00FF}] \]
\begin{itemize}
    \item \texttt{[...]}: Spécifie une classe de caractères, correspondant à n'importe quel caractère unique inclus entre parenthèses.
    \item \texttt{\^{}}: Entre parenthèses de classe de caractères, cela annule la classe, donc elle correspond à n'importe quel caractère \emph{non} répertorié entre parenthèses.
    \item \texttt{\textbackslash u0000-\textbackslash u00FF}: Définit une plage de caractères Unicode de \( U+0000 \) à \( U+00FF \), qui comprend à la fois les blocs Basic Latin et Latin-1 Supplement.
\end{itemize}

\newpage
\subsubsection{Logique de méthode}
\noindent
La méthode \texttt{re.search(r"[\textbackslash u0000-\textbackslash u00FF]", s)} recherche dans la chaîne \( s \), recherchant tout caractère en dehors de plage \( U+0000 \) jusqu'à \( U+00FF \):
\begin{itemize}
    \item Si un tel caractère est trouvé, \texttt{re.search} renvoie un objet de correspondance, qui est évalué à \texttt{True}.
    \item Si aucun caractère de ce type n'est trouvé (c'est-à-dire que tous les caractères sont dans la plage spécifiée), il renvoie \texttt{None}, qui est évalué à \texttt{False}.
\end{itemize}
L'utilisation de l'opérateur \texttt{not} inverse le résultat de \texttt{re.search}. Ainsi, la méthode renvoie \texttt{False} si des caractères se trouvent en dehors de la plage Unicode spécifiée, et \texttt{True} si tous les caractères s'y trouvent.

\noindent
La figure ~\ref{fig:islatin} illustre le code nécessaire pour cette méthode.

\begin{figure}[H]
	\centering
	\includegraphics[width=1\textwidth]{logos/islatin.png}
	\caption{Code de méthode \_\_is\_latin}
	\label{fig:islatin}
\end{figure}
